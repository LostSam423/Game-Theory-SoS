\begin{flushleft}
\section{Introduction}
\end{flushleft}
\subsection{Game Theory}
In a game as simple as Rock, Paper, Scissors, when two people play their respective moves at a time, they hope to play the move that is in favour of them and eventually win the game. The strategy applied by both the users, like looking for a pattern in their opponents' previous moves to predict their next move, is what makes up a part of this topic. Largely popularised by the famous movie \textit{A Beautiful Mind}, based on the life of Nobel winning laureate \textbf{John Nash}, Game Theory is essentially the science of strategy, or at least the optimal decision-making of independent and competing actors in a strategic setting.
\subsection{Basic Terminologies}
	\begin{itemize}
	\item \textbf{Players:}\\
	The strategic decison-makers in the context of the game. These can be as small as individuals and as large as governments or multi-national companies.
	\item \textbf{Rational:}\\
		An individual is considered \textit{rational} if she has well defined objectives (or preferences) over the set of possible outcomes and she implements the best available \textit{strategy} to pursue them. In reality, assumption of rationality might be an unrealistic one. These limitations is what gives birth to the concept of \textit{bounded rationality} which is an active area of research currently.
	\item \textbf{Strategy:}\\
	A proper set of action plans chosen by a player in a certain setting, whose outcome depends not only on her action, but on others' too.
	\item \textbf{Rules:}\\
	A set of statements that clarifies, demarcates and/or interprets the proceedings of a game.  
	\item \textbf{Actions:}\\
	These are the choices available to the player from which she has to choose.
	\item \textbf{Payoff:}\\
	Sounds like a reward, it acts as a motivating factor behind the actions of the players and the reason for their participation.
	\item \textbf{Common Knowledge:}\\
	As we consider all players in the game to be \textit{rational}, everyone of the players knows about the model, everybody knows that everybody knows about the model, everybody knows that everybody knows that everybody knows it, and so on.
	\end{itemize}